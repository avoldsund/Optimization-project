For proving that the problem \eqref{eq:system5} admits exactly one solution, we use Proposition 3 in \cite{intOptCon}. $\Omega$ is both non-empty and closed, as previously shown. As the function in Proposition 3 is quadratic, the proposition will work for our case. As the requirements are fulfilled, the proposition tells us that there is at least one globally optimal solution to \eqref{eq:system5}, but the solution is unique as well, since $\Omega$ is convex.

By setting $\bm{f}^{\textrm{ext}} = \bm{0},$ we can find this solution. Looking at \eqref{eq:system5}, we see that $g(\bm{q})$ is minimized if $\bm{q}$ is equal to $\bm{0}$. Hence it is easy to see that this implies that $\bm{f}^{\textrm{supp}} = \bm{0}$ as well, from the constraints. This gives $(\bm{q},\bm{f}^{\textrm{supp}}) = (\bm{0},\bm{0})$ which is our globally optimal solution.



% USIKKER PÅ DETTTTTTTEEEEEEEEEEEEEEEEEEEEEEEEEEEEEEEEEEEEE!