It is possible to use Theorem 2 \cite{optThe} for system \eqref{eq:system6} as well. Whenever feasible means that $\Omega$ is non-empty and since the constraints have no strict inequalities, we know $\Omega$ is closed. What remains to show is that $g(\bm{A},\bm{q})$ is coercive and lower semi-continuous. 

First we check if it is coercive. As $g(\bm{A},\bm{q})$ is a function of two variables, we must check if it is coercive when both $\vectornorm{\bm{A}}$ and $\vectornorm{\bm{q}}$ goes to infinity. Since $\bm{q}$ is dependent on $\bm{f}^{\textrm{supp}}$, and $\vectornorm{\bm{f}^{\textrm{supp}}}$ goes to infinity, we know it's no problem that $\vectornorm{\bm{q}}$ does it as well.  We divide it in two cases, when the element in $\bm{q}$ and $\bm{A}$ that goes to infinity, namely $q_{k}$ and $A_{p}$ are equal indices, or different indices. We will start with the case where k = p. Notice that in the last limit, $\vectornorm{\bm{q}}_\infty = q_{k}$, $\vectornorm{\bm{A}}_\infty = q_{p}$.

\begin{equation}
\begin{aligned}
\textrm{k} = \textrm{p}:
\lim_{\substack{\vectornorm{\bm{q}}\to\infty \\ \vectornorm{\bm{A}}\to\infty \\ \vectornorm{\bm{f}^ {\textrm{supp}}}\to\infty}} \frac{1}{2}\sum_{j=1}^{m}\frac{\ell_{j}q_{j}^{2}}{E_{j}A_{j}} =  \frac{1}{2}\sum_{\substack{j=1 \\ j \neq k}}^{m}\frac{\ell_{j}q_{j}^{2}}{E_{j}A_{j}} + 
\lim_{\substack{{q}_{k}\to\infty \\ {A}_{k}\to\infty \\ f^{\textrm{supp}_{k}}\to\infty}}  \frac{\ell_{k}{q_{k}}^{2}}{E_{k} A_{k}} = \infty.
\end{aligned}
\end{equation}

\begin{equation}
\begin{aligned}
\textrm{k} \neq \textrm{p}:
\lim_{\substack{\vectornorm{\bm{q}}\to\infty \\ \vectornorm{\bm{A}}\to\infty \\ \vectornorm{\bm{f}^ {\textrm{supp}}}\to\infty}} \frac{1}{2}\sum_{j=1}^{m}\frac{\ell_{j}q_{j}^{2}}{E_{j}A_{j}} =  \frac{1}{2}\sum_{\substack{j=1 \\ j \neq k \\ j \neq p}}\frac{\ell_{j}q_{j}^{2}}{E_{j}A_{j}} + 
\lim_{\substack{{q}_{k}\to\infty \\ \vectornorm{\bm{f}^{\textrm{supp}}}\to\infty}}  \frac{\ell_{k}{q_{k}}^{2}}{E_{k} A_{k}}  + \lim_{{A}_{p}\to\infty}  \frac{\ell_{p}{q_{p}}^{2}}{E_{p} A_{p}} = \infty.
\end{aligned}
\end{equation}

Lastly, $g(\bm{A},\bm{q})$ is lower semi-continuous because it is continuous. Hence theorem 2 applies, and we have at least one global minimum.

We now check to see if the theorem still holds when $\underline{A}_{j} = 0$. The constraint now becomes $A_{j} \geq 0$. If $q_{j} \neq 0$ and $\frac{q_{j}^2}{0} = +\infty$ we see that $g(0,q)$ is equal to infinity. If, on the other hand, $q_{j} = 0$, we get $\frac{0}{0}$ inside the sum, which is equal to 0. Hence we get a value $\rho \neq 0$. We see that the first case is still continuous, while the second is lower semi-continuous, as it drops from $\infty$ when $q_{j}$ goes to 0, while it is $\rho$ when it's equal to 0. Thus theorem 2 still applies.

%\begin{equation}
%\frac{1}{2}\sum_{j=1}^{m}\frac{\ell_{j}q_{j}^{2}}{E_{j}A_{j}} = \frac{1}%{2}\sum_{\substack{j=1 \\ j \neq k}}^{m}\frac{\ell_{j}q_{j}^{2}}%{E_{j}A_{j}} = \rho
%\end{equation}



