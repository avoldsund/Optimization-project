
% DETTE ER UFERDIG
The KKT-conditions for \eqref{eq:system6} are as follows:
\begin{equation}
\label{KKT_system7}
\begin{aligned}
\bm{Dq} = \bm{B}^{T}\bm{\lambda}\\
\bm{I}^{T}_{\textrm{supp}}\bm{\lambda} = 0 \\
\bm{K} + \frac{\mu}{M-\sum\limits_{j=1}^m\rho_{j}l_{j}A_{j}}\bm{T} - \mu\bm{C} = 0 \\
\bm{Bq} - \bm{I}_{\textrm{supp}}\bm{f}^{\textrm{supp}} - \bm{I}_{\textrm{ext}}\bm{f}^{\textrm{ext}} = 0 \\
(\bm{Bq} - \bm{I}_{\textrm{supp}}\bm{f}^{\textrm{supp}} - \bm{I}_{\textrm{ext}}\bm{f}^{\textrm{ext}})^T\bm{\lambda} = 0 \\
\bm{K} + \bm{T}\phi + \bm{\gamma} - \bm{\xi} = 0 \\
\phi \geq 0  \\
\gamma_j \geq 0 \text{ for } j = 1,...,m \\
\xi_j \geq 0 \text{ for } j = 1,...,m \\
\phi(M-\sum\limits_{j=1}^m\rho_{j}l_{j}A_{j}) = 0 \\
\gamma_{j}(A_{j}-\overline{A_{j}}) = 0 \text{ for } j = 1,...,m \\
\xi_{j}(\underline{A_{j}}-A_{j}) = 0 \text{ for } j = 1,...,m \\
\end{aligned}
\end{equation}

\begin{align*}
&\bm{C} = \begin{bmatrix}
\frac{1}{A_1-\underline{A_{1}}} - \frac{1}{\overline{A_{1}}-A_{1}}, \hdots, \frac{1}{A_m-\underline{A_{m}}} - \frac{1}{\overline{A_{m}}-A_{m}}
\end{bmatrix}^T
\end{align*}
By comparing the two systems, we see that the approximations to the Lagrange multipliers corresponding to the inequality constraints in (\ref{KKT_system6}), in terms of an optimum for (\ref{KKT_system7}), must satisfy the following:

\begin{align*}
\frac{\mu}{M-\sum\limits_{j=1}^m\rho_jl_jA_j}
\begin{bmatrix}
\rho_1l_1 \\
\vdots \\
\rho_ml_m\\
\end{bmatrix} &= \phi
\begin{bmatrix}
\rho_1l_1 \\
\vdots \\
\rho_ml_m\\
\end{bmatrix}\\
\begin{bmatrix}
\gamma_1 \\
\vdots \\
\gamma_m\\
\end{bmatrix} &= \mu
\begin{bmatrix}
\frac{1}{\overline{A}_1-A_1} \\
\vdots \\
\frac{1}{\overline{A}_m-A_m}\\
\end{bmatrix}\\
\begin{bmatrix}
\xi_1 \\
\vdots \\
\xi_m\\
\end{bmatrix} 
&= \mu
\begin{bmatrix}
\frac{1}{A_1-\underline{A}_1} \\
\vdots \\
\frac{1}{A_m-\underline{A}_m}
\end{bmatrix}.
\end{align*}
From this we conclude that
\begin{align*}
\phi = \frac{\mu}{M-\sum\limits_{j=1}^m\rho_jl_jA_j}\\
\gamma_j = \frac{\mu}{\overline{A}_j-A_j} \text{ for } j = 1,..,m \\
\xi_j = \frac{\mu}{A_j-\underline{A}_j} \text{ for } j = 1,..,m.
\end{align*}

Since $g(\bm{A},\bm{q},\bm{f}^{supp})$ have been permuted from the previous objective function with function only dependent on $\bm{A}$, we only need to consider the top left part of the Hessian changing the positive semi-definiteness. We can split the top left corner of the Hessian into two parts, where $\bm{K}$ is the same as before:
\begin{align*}
& \bm{\bigtriangledown}_{A}^2 g(\bm{A},\bm{q},\bm{f}^{supp}) = \bm{K} + U(\bm{A}) \\
U(\bm{A}) = \frac{\mu}{(M-\sum\limits_{j=1}^{m}\rho_jl_jA_j)^2} &
\begin{bmatrix}
(\rho_1l_1)^2 & (\rho_1l_1)(\rho_2l_2) & \hdots \\
(\rho_1l_1)(\rho_2l_2) & (\rho_2l_2)^ 2 & \hdots \\
\vdots & \vdots & \ddots \\
\end{bmatrix} + \\
& \mu 
\begin{bmatrix}
\frac{1}{(A_1-\underline{A_{1}})^2} + \frac{1}{(\overline{A_{1}}-A_{1})^2} &  & \text{\large{0}} \\
  & \ddots &  \\
\text{\large0} &  & \ddots \\
\end{bmatrix}
\end{align*}
We want to compute $\bm{w}_{A}^T\bm{\bigtriangledown}_{A}^2g(\bm{A},\bm{q}\bm{f}^{supp})\bm{w}_A$ for some component $\bm{w}_A$ of $w\epsilon\Omega$. Since $\bm{w}_A^T\bm{K}\bm{w}_A$ is an element of the previous version, it does not contribute to make the matrix not positive semi-definite. Therefore we only compute $\bm{w}_A^TU(\bf{A})\bm{w}_A$. It can be shown that:
\begin{align*}
\bm{w}_A^TU(\bm{A})\bm{w}_A = \frac{(\sum\limits_{j=1}^m\rho_jl_jA_j)^2}{(M-\sum\limits_{j=1}^m\rho_jl_jA_j)^2} + \mu\sum\limits_{j=1}^mA_j^2(\frac{1}{(A_j-\underline{A_{j}})^2} + \frac{1}{(\overline{A_{j}}-A_{j})^2}).
\end{align*}
Since all the terms are bigger than or equal to zero, we conclude that \\ $\bm{w}_A^TU(\bm{A})\bm{w}_A \geq 0$, and therefore that $\bm{w}^T\bm{\bigtriangledown}^2 g(\bm{A},\bm{q},\bm{f}^{supp})\bm{w} \geq 0$ for all $\bm{w}\epsilon\Omega$. Thus the Hessian is positive semi-definite, and  $g(\bm{A},\bm{q},\bm{f}^{supp})$ is convex. Since our constraints are affine, by \cite{KKT_suff} we can conclude that the KKT-conditions are both necessary and sufficient for optimality.


